\chapter{Introduction}
\label{introduction}

Single Non-Transferable vote is the most common method of voting used in the United States. It seems very intuitive; everyone gets one vote that they cast for one person.

Although you can say that this method is fair in one aspect, that fairness breaks down given more than two choices, and continues to worsen as the field of options expands. By having more choices to cast your vote towards, people with similar preferences given multiple options clustered in likeness can have their votes split between the choices, diluting their opinion and possibly losing to a less-preferred option overall.

Ranked-Choice voting, also known as Single Transferable Vote, is an attempt to solve this. Voters ranks the available options on their ballot, and these ballots are sorted to candidates by first choice. If a candidate has a majority of first place votes in the current field, they win. If no one has this majority, whomever is presently in last place is eliminated, and their current vote pool is redistributed by the following ranking. This process is repeated until a candidate has a majority of the outstanding votes among the field.



\begin{center}
 \begin{tabular}{|c|c c c c c c|} 
 \hline
 Rank & 4 & 7 & 2 & 8 & 9 & 1 \\ [0.5ex] 
 \hline\hline
 1 & A & A & B & B & C & C \\ 
 \hline
 2 & B & C & A & C & A & B \\
 \hline
 3 & C & B & C & A & B & A \\
 \hline
\end{tabular}
\end{center}

\begin{center}
 \begin{tabular}{|c|c|c c c c c c|} 
 \hline
 Rank & Weight & 4 & 7 & 2 & 8 & 9 & 1 \\ [0.5ex] 
 \hline\hline
 1 & 3 & A & A & B & B & C & C \\ 
 \hline
 2 & 1 & B & C & A & C & A & B \\
 \hline
 3 & 0 & C & B & C & A & B & A \\
 \hline
\end{tabular}
\end{center}