\chapter{Definitions}
\label{AppendixA}

\section{Preference}
This is the will of a constituent to want a candidate over another candidate. This can include both the ordinal candidate rankings, as well as intensity. For example, two people might rank candidate A above candidate B, but one person could significantly prefer A over B while the other is nearly indifferent.

\section{Mean Preference}
This is the average satisfaction that a total constituency feels towards a candidate. For example, suppose there is a candidate that is loved by all; they would have a high mean preference. Likewise, a candidate who is hated by all would have a low mean preference. Someone who is loved by half of a constituency and hated by the other half would have a moderate mean preference, and this would be equal to the mean preference of a candidate who is hated/loved by none with lukewarm sentiment by all.

If this were to be calculated via honest Range voting on a scale of 1 to 10, low mean preference would result in an average score closer to 0, high mean preference would result in an average score close to 10, and moderate mean preference would be close to 5.

\section{Median Preference}
This is the satisfaction that moderate constituents feel towards a candidate. However, it ought to be noted that moderate here is not reserved to a two-party system and instead references the middle portion of voters sorted by intensity from the votes they received.

Suppose we have candidate A and candidate B. A receives many low 2 value votes with some 9s and 10s to boost them. B receives a linear progression of 3, 4, 5, 6, 7 per quintile from the voters. In both instances, the mean is 5, but candidate B has the higher mean preference at 5 instead of A’s 2.
